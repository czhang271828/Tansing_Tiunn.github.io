\documentclass{MainStyle}

\usepackage{amsthm, amsfonts, amsmath, amssymb, quiver, mathrsfs, newclude, tikz-cd, ctex}

% Customise href Colours.
\usepackage[colorlinks = true,
            linkcolor = blue,
            urlcolor  = blue,
            citecolor = blue,
            anchorcolor = blue]{hyperref}

\newcommand{\changeurlcolor}[1]{\hypersetup{urlcolor=#1}}       

\newcommand*{\name}{张陈成}
\newcommand*{\id}{023071910029}
\newcommand*{\course}{$K$-理论笔记}
\newcommand*{\assignment}{Sylvester 理论}

\theoremstyle{definition}
\newtheorem{example}{例}

\theoremstyle{definition}
\newtheorem{slogan}{原旨}

\theoremstyle{definition}
\newtheorem{definition}{定义}

\theoremstyle{definition}
\newtheorem{proposition}{命题}

\theoremstyle{definition}
\newtheorem{problem}{问题}

\theoremstyle{definition}
\newtheorem{assumption}{假定}

\theoremstyle{definition}
\newtheorem{theorem}{定理}

\theoremstyle{remark}
\newtheorem{remark}{注}

\theoremstyle{remark}
\newtheorem{lemma}{引理}
\allowdisplaybreaks

\begin{document}
\maketitle
\tableofcontents

\section{Sylvester 秩函数}

\begin{proposition}[有限展示模]
    给定模正合列 $0\to K\to Y\to Z\to 0$, 试回顾以下关于有限展示模之论断.
    \begin{enumerate}
        \item 若 $K$ 有限生成, $Z$ 有限生成, 则 $Y$ 有限生成.
        \item[1'.] 若 $Y$ 有限生成, 则 $Z$ 有限生成, 但 $K$ 未必.
        \item 若 $Y$ 有限生成, $Z$ 有限展示, 则 $K$ 有限生成.
        \item 若 $K$ 有限生成, $Y$ 有限展示, 则 $Z$ 有限展示.
        \item 若 $K$ 有限展示, $Z$ 有限展示, 则 $Y$ 有限展示.
    \end{enumerate}
\end{proposition}

\begin{definition}[Sylvester 秩函数]
    定义环 $R$ 上的 Sylvester 函数为 $\rho: \text{有限展示 $R$-模}\,\to \mathbb R_{\geq 0}$, 满足如下条件.
    \begin{enumerate}
        \item 归一化条件: $\rho(R)=1$.
        \item $\rho$ 可加, 即, $\rho(X\oplus Y)=\rho(X)+\rho(Y)$ 对一切有限展示模成立.
        \item 对正合列 $X\to Y\to Z\to 0$, 总有 $\rho(Z)\leq \rho(Y)\leq \rho(X)+\rho(Z)$.
    \end{enumerate}
\end{definition}

\begin{remark}
    Sylvester 秩函数衡量有限生成模正合列间的有限展示程度. 置 $X=\mathrm{ker}(Y\to Z)$, 则
    \begin{enumerate}
        \item $\rho(X)+\rho(Z)\geq \rho(Y)$ 对应论断: 若 $X$ 与 $Z$ 有限展示, 则 $Y$ 亦然.
        \item $\rho(Y)\geq\rho(Z)$ 对应论断: 若 $Y$ 有限展示, 则 $Z$ 亦然.
    \end{enumerate}
\end{remark}

\begin{proposition}[Sylvester 秩函数的等价定义]
    环 $R$ 的 Sylvester 秩函数等价定义亦可由
    \begin{align*}
        \rho':\text{有限生成投射 $R$ 模间态射}\,\to \mathbb R_{\geq 0}
    \end{align*}
    以及以下四条法则等价地给出.
    \begin{enumerate}
        \item 归一化条件 $\rho'(\mathrm{id}_R)=1$.
        \item $\rho'(\mathrm{diag}(f,g))=\rho'(f)+\rho'(g)$.
        \item $\rho'\begin{pmatrix}
                      f & h \\0&g
                  \end{pmatrix}\geq\rho'\begin{pmatrix}
                      f & 0 \\0&g
                  \end{pmatrix}=\rho'(f)+\rho'(g)$.
        \item $\rho'(fg)\leq \min (\rho'(f),\rho'(g))$. 特别地, $\rho'(f)=\rho'(f\varphi)=\rho'(\psi f)$ 若 $\varphi$ 与 $\psi$ 为同构.
    \end{enumerate}
    其转换关系如下. 对正合列 $\bullet\overset f\to Q\overset c\to C\to 0$, 有
    \begin{align*}
        \rho(C)+\rho'(f)=\rho(Q)=\rho'(\mathrm{id}_Q)
    \end{align*}
\end{proposition}

\begin{remark}
    今后不区分 $\rho$ 与 $\rho'$. 换言之, $\rho$ 同时定义在有限展示模与投射模间态射上.
\end{remark}

\begin{proposition}[Sylvester 定理]
    对任意域上同阶数的方阵 $A$ 与 $B$, 总有 $\mathrm{rank}(AB)\geq \mathrm{rank}(A)+\mathrm{rank}(B)-n$.\footnote{线性代数中常用定理.}
\end{proposition}

\begin{proposition}[Sylvester 定理]
    对任意映射链 $P''\overset f\to P\overset g\to P'$, 总有 $\rho(gf)\geq \rho(f)+\rho(g)-\rho(\mathrm{id}_P)$.
    \begin{proof}
        显然 $\rho(f)+\rho(g)=\rho\begin{pmatrix}
                f & \\&g
            \end{pmatrix}\leq \rho \begin{pmatrix}
                f & \mathrm{id}_P \\&g
            \end{pmatrix}$. 依照 $\rho(-)$ 的同构不变性, 以及列初等变换
        \begin{align*}
            \begin{pmatrix}
                f & \mathrm{id}_P \\&g
            \end{pmatrix}
            \sim
            \begin{pmatrix}
                 & \mathrm{id}_P \\-gf&g
            \end{pmatrix}
            \sim
            \begin{pmatrix}
                 & \mathrm{id}_P \\-gf&g
            \end{pmatrix}
            \sim
            \begin{pmatrix}
                \mathrm{id}_P \\&gf
            \end{pmatrix}
        \end{align*}
        从而 $\rho(f)+\rho(g)\leq \rho(fg)+\rho(\mathrm{id}_P)$.
    \end{proof}
\end{proposition}

\begin{example}
    给定单 Artin 环 $M_n(D)$\footnote{依照 Artin-Wedderbrun 定理, 半单 Artin 代数形如除环之矩阵环代数之积.}, 则 $\rho(P)=\dfrac{l(P)}n$. 因此单 Artin 环上的秩函数即长度单位化.
\end{example}

\section{(广义)局部化}

\section{$R$-代数到单 Artin 代数的态射类}

\section{三角范畴及更多}
\end{document}