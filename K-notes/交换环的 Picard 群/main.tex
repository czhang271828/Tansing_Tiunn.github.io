\documentclass{MainStyle}

\usepackage{amsthm, amsfonts, amsmath, amssymb, quiver, mathrsfs, newclude, tikz-cd, ctex}

% Customise href Colours.
\usepackage[colorlinks = true,
            linkcolor = blue,
            urlcolor  = blue,
            citecolor = blue,
            anchorcolor = blue]{hyperref}

\newcommand{\changeurlcolor}[1]{\hypersetup{urlcolor=#1}}       

\newcommand*{\name}{张陈成}
\newcommand*{\id}{023071910029}
\newcommand*{\course}{$K$-理论笔记}
\newcommand*{\assignment}{交换环的 Picard 群}

\theoremstyle{definition}
\newtheorem{example}{例}

\theoremstyle{definition}
\newtheorem{slogan}{原旨}

\theoremstyle{definition}
\newtheorem{definition}{定义}

\theoremstyle{definition}
\newtheorem{proposition}{命题}

\theoremstyle{definition}
\newtheorem{problem}{问题}

\theoremstyle{definition}
\newtheorem{assumption}{假定}

\theoremstyle{definition}
\newtheorem{theorem}{定理}

\theoremstyle{remark}
\newtheorem{remark}{注}

\theoremstyle{remark}
\newtheorem{lemma}{引理}
\allowdisplaybreaks

\begin{document}
\maketitle

\section{交换环的 Picard 群}

\begin{theorem}
    给定交换环 $R$ 与连续函数 $f:\mathrm{Spec}(R)\to X$, 其中 $X$ 具备离散拓扑. 存在分解 $R\simeq \prod R_i$ 使得 $\coprod\mathrm{Spec}(R_i)\simeq \mathrm{Spec}(R)$, 且 $f|_{\mathrm{Spec}(R_i)}$ 为常映射.
    \begin{proof}

    \end{proof}
\end{theorem}

\begin{definition}[可逆模]
    称 $M$ 是交换环 $R$ 上有限生成的模. 称 $M$ 可逆, 若以下等价命题成立.
    \begin{enumerate}
        \item 存在 $R$-模 $N$ 使得 $M\otimes N\simeq R$, 且 $M\simeq \mathrm{Hom}_R(N,R)$.
        \item $M\otimes_R-$ 为 $R$-模范畴到自身的等价.
        \item $M$ 是有限生成的秩恒为 $1$ 的投射模.
    \end{enumerate}
    实际上有 $\mathrm{Hom}_R(N,R)\simeq M$.
\end{definition}

\begin{definition}[Picard 群]
    记环 $R$ 中 Picard 群为 $\mathrm{Pic}(R)$ 有限生成可逆模 $\langle M\rangle$ 构成的乘法群. 其中
    \begin{enumerate}
        \item $\langle M\otimes_R N\rangle=\langle M\rangle\cdot \langle  N\rangle$.
        \item $\langle \mathrm{Hom}_R(M,R)\rangle=\langle M\rangle^{-1}$.
        \item $\langle R\rangle$ 为乘法单位.
    \end{enumerate}
\end{definition}

\begin{remark}
    $\mathrm{Pic}:\mathrm{Ring}\to \mathrm{Ab}$ 为(协变)函子. 特别地,
    \begin{align*}
        \mathrm{Pic}:\left[R\overset f\longrightarrow S\right]\mapsto [P\mapsto S\otimes_R P].
    \end{align*}
\end{remark}

\end{document}