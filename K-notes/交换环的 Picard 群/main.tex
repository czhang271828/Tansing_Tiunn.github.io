\documentclass{MainStyle}

\usepackage{amsthm, amsfonts, amsmath, amssymb, quiver, mathrsfs, newclude, tikz-cd, ctex}

% Customise href Colours.
\usepackage[colorlinks = true,
            linkcolor = blue,
            urlcolor  = blue,
            citecolor = blue,
            anchorcolor = blue]{hyperref}

\newcommand{\changeurlcolor}[1]{\hypersetup{urlcolor=#1}}       

\newcommand*{\name}{张陈成}
\newcommand*{\id}{023071910029}
\newcommand*{\course}{$K$-理论笔记}
\newcommand*{\assignment}{交换环的 Picard 群}

\theoremstyle{definition}
\newtheorem{example}{例}

\theoremstyle{definition}
\newtheorem{slogan}{原旨}

\theoremstyle{definition}
\newtheorem{definition}{定义}

\theoremstyle{definition}
\newtheorem{proposition}{命题}

\theoremstyle{definition}
\newtheorem{problem}{问题}

\theoremstyle{definition}
\newtheorem{assumption}{假定}

\theoremstyle{definition}
\newtheorem{theorem}{定理}

\theoremstyle{remark}
\newtheorem{remark}{注}

\theoremstyle{remark}
\newtheorem{lemma}{引理}
\allowdisplaybreaks

\begin{document}
\maketitle

\section{可逆模 $=$ 线丛 $=$ 交换半环 $R\mathrm{-Mod}$ 中单位}

\begin{definition}[有限生成投射模的秩函数]
    kaplansky 定理表明局部环上的投射模自由, 其交换且有限生成之情形已在前文证明(中山引理之推论). 今给定交换环 $R$, 定义有限生成投射模 $P$ 的秩函数为
    \begin{align*}
        \mathrm{rank}_P:\mathrm{spec}(R)\to \mathbb N,\quad \mathfrak p\mapsto \mathrm{rank}_{R_{\mathfrak p}}(P_{\mathfrak p}).
    \end{align*}
    简而言之, $\mathrm{rank}_P(\mathfrak p)$ 是 $R$-模 $P$ 在 $\mathfrak p$-局部化下(作为自由 $R_{\mathfrak p}$-模)的秩.
\end{definition}

\begin{proposition}[秩函数的乘法]
    给定交换环 $R$ 与有限生成投射 $R$-模 $M$ 与 $N$, 则有
    \begin{align*}
        \mathrm{rank}_{M\otimes N}(\mathfrak p)\mapsto \mathrm{rank}_M(\mathfrak p)\cdot \mathrm{rank}_N(\mathfrak p)\quad (\forall \mathfrak p\in \mathrm{spec}(R)).
    \end{align*}
    \begin{proof}
        此处投射模的张量积仍投射. 考虑 $M\oplus M'\simeq R^\lambda$, 则有
        \begin{align*}
            (M\otimes N)\oplus (M'\otimes N)\simeq R^\lambda\otimes N \simeq N^\lambda.
        \end{align*}
        从而 $M\otimes N$ 仍为自由模的直和项, 因此投射. 假定 $M_{\mathfrak p}\simeq R_{\mathfrak p}^m$ 以及 $N_{\mathfrak p}\simeq R_{\mathfrak p}^n$, 则
        \begin{align*}
            M_{\mathfrak p}\otimes_{R_{\mathfrak p}} N_{\mathfrak p}\simeq R_{\mathfrak p}^m\otimes_{R_{\mathfrak p}} R_{\mathfrak p}^n\simeq R_{\mathfrak p}^{mn}.
        \end{align*}
    \end{proof}
\end{proposition}

\begin{proposition}[对偶模回顾]
    给定交换环 $R$, 定义 $X\in \mathsf{Ob}(R\mathrm{-Mod})$ 的对偶模为
    \begin{align*}
        X^\ast:=\mathrm{Hom}_{R\mathrm{-Mod}}(X,R)\quad \in \mathsf{Ob}(R^{\mathrm{op}}\mathrm{-Mod}).
    \end{align*}
    有以下关于对偶模的常用性质.
    \begin{enumerate}
        \item $\varepsilon: P\to P^{\ast\ast}$ 为典范单态射.
        \item 自由模与投射模的一种等价定义如下.
              \begin{itemize}
                  \item $F$ 是\textbf{自由 $R$-模}, 当且仅当存在指标集 $I$ 与 $\{(x_i,f_i)\in F\times F^\ast\}_{i\in I}$ 使得有分解(有限和) $\displaystyle x=\sum _{i\in I}f_i(x)x_i$, 且有限和 $\displaystyle x=\sum _{i\in I}a_i x_i$ 对一切 $x\in F$ \textbf{唯一}.
                  \item $P$ 是\textbf{投射 $R$-模}, 当且仅当存在指标集 $I$ 与 $\{(x_i,f_i)\in P\times P^\ast\}_{i\in I}$ 使得有分解(有限和) $\displaystyle x=\sum _{i\in I}f_i(x)x_i$, 但有限和 $\displaystyle x=\sum _{i\in I}a_i x_i$ 对 $x\in P$ \textbf{不必唯一}.
              \end{itemize}
        \item 有限生成投射模的对偶模同为投射 $R$-模. 具体地, 对任意 $P\oplus Q\simeq R^n$ 总有
              \begin{align*}
                  \mathrm{Hom}_R(P,R)\oplus \mathrm{Hom}_R(Q,R)\simeq \mathrm{Hom}_R(R^n,R)\simeq (\mathrm{End}_R(R))^n\simeq R^n.
              \end{align*}
              此时 $\varepsilon: P^{\ast\ast}\simeq P$ 为同构.
    \end{enumerate}
\end{proposition}

\begin{proposition}[有限生成投射模之对偶不改变秩函数]
    取 $R$ 上有限生成投射模 $P$, 则有
    \begin{align*}
        P_\mathfrak p^n\simeq R_\mathfrak p^n\simeq \mathrm{Hom}_{R_\mathfrak p}(P_\mathfrak p,R_\mathfrak p)^n\simeq (P^\ast_\mathfrak p)^n\simeq (P^\ast)_\mathfrak p^n.
    \end{align*}
\end{proposition}

\begin{definition}[$R$-模范畴的半环结构]
    $(R\mathrm{-Mod},\oplus, \otimes)$ 为交换半环, 即,
    \begin{enumerate}
        \item 环中元素为 $\mathsf{Ob}(R\mathrm{-Mod})/\simeq$. 为方便记号, 今后省略商关系.
        \item $(R\mathrm{-Mod},\oplus)$ 为交换幺半群, 其幺元为 $0$;
        \item $(R\mathrm{-Mod}, \otimes)$ 为交换幺半群, 其幺元为 $R$;
        \item $\oplus$ 与 $\otimes$ 分别作为加法与乘法, 满足分配律.
    \end{enumerate}
\end{definition}

\begin{definition}[可逆模(线丛)]
    交换半环 $(R\mathrm{-Mod},\oplus, \otimes)$ 中的单位(可逆乘法元)全体为\textbf{可逆模(线丛)}.
\end{definition}

\begin{proposition}
    取交换环 $R$ 上有限生成模 $M$. 称 $M$ 可逆, 若以下等价命题成立.
    \begin{enumerate}
        \item 存在 $R$-模 $N$ 使得 $M\otimes N\simeq R$. 换言之, $M$ (所属的同构类)是环 $(R\mathrm{-Mod},\oplus,\otimes)$ 中的乘法逆元.
        \item $M\otimes_R-$ 为 $R$-模范畴到自身的等价. 换言之, 存在 $N\otimes-$ 使得 $(M\otimes N\otimes -)\simeq (R\otimes -)$.
        \item $M$ 是有限生成的秩恒为 $1$ 的投射模.
    \end{enumerate}
    若前两则成立, 则可取 $\mathrm{Hom}_R(N,R)\simeq M$.
    \begin{proof}
        $1\Longleftrightarrow 2$ 是显然的. 函子 $-\otimes M$ 给出范畴 $R\mathrm{-Mod}$ 到自身的范畴等价, 当且仅当 $M$ 是环 $(R\mathrm{-Mod},\oplus,\times)$ 的乘法可逆元. 换言之, 存在 $N$ 使得 $N\otimes M\simeq R\simeq M\otimes N$. \par
        $3\implies 1$ 若 $M$ 是秩 $1$ 的投射模, 记 $N=M^\ast$, 并考虑赋值映射
        \begin{align*}
            M\otimes N\to R,\quad x\otimes f\mapsto f(x).
        \end{align*}
        显然该映射对任意素理想的局部化是同构. 由于 $M$ 投射, 从而对任意素理想 $\mathfrak p$,
        \begin{align*}
            R_{\mathfrak p}\simeq \mathrm{Hom}_{R_{\mathfrak p}}(M_{\mathfrak p},R_{\mathfrak p})\simeq (\mathrm{Hom}_R(M,R))_{\mathfrak p}\simeq N_{\mathfrak p}.
        \end{align*}
        因此 $N=M^\ast$ 也是秩为 $1$ 的投射模. 由于 $M_{\mathfrak p}\otimes_{R_\mathfrak p} N_{\mathfrak p}\simeq R_{\mathfrak p}$ 对一切 $\mathfrak p$ 成立, 从而 $M\otimes N\simeq R$.\par
        $1\implies 3$ 在 $M\otimes N\simeq R$ 两端作 $\mathfrak p$-局部化, 则 $\mathrm{rank}_M=\mathrm{rank}_N=\mathrm{rank}_R$. 下仅需证明 $M$ 与 $N$ 投射. 记 $M$ 的有限生成元 $\{m_i\}_{1\leq i\leq m}$ 与一组 $\{n_i\}_{1\leq i\leq m}\subseteq N$ 使得有
        \begin{align*}
            \sum_{1\leq i\leq m}m_i\otimes n_i =1\otimes 1\quad \left(\simeq 1\in R\right).
        \end{align*}
        以此构造满同态 $N^m\twoheadrightarrow R$, 其中 $(x_1,\ldots x_m)\mapsto \sum_{1\leq i\leq m}m_i\otimes x_i$. 显然该满同态可裂, 记
        \begin{align*}
            N^m\simeq R^m\otimes N\simeq R\oplus Q.
        \end{align*}
        从而 $R^m\simeq R^m\otimes (M\otimes N)\simeq N^m\otimes N\simeq (R\oplus Q)\otimes N\simeq N\oplus (Q\otimes N)$. 因此 $N$ 投射. 对称地, $M$ 亦投射.
    \end{proof}
\end{proposition}

\begin{definition}[Picard 群]
    记环 $R$ 中 Picard 群为 $\mathrm{Pic}(R)$ 有限生成可逆模 $\langle M\rangle$ 构成的乘法群. 其中
    \begin{enumerate}
        \item $\langle M\otimes_R N\rangle=\langle M\rangle\cdot \langle  N\rangle$.
        \item $\langle \mathrm{Hom}_R(M,R)\rangle=\langle M\rangle^{-1}$.
        \item $\langle R\rangle$ 为乘法单位.
    \end{enumerate}
\end{definition}

\begin{remark}
    $\mathrm{Pic}:\mathrm{Ring}\to \mathrm{Ab}$ 为(协变)函子. 特别地,
    \begin{align*}
        \mathrm{Pic}:\left[R\overset f\longrightarrow S\right]\mapsto [P\mapsto S\otimes_R P].
    \end{align*}
    结合律与单位律由张量积的结合律保证.
\end{remark}

\section{一些代数几何解释}

\end{document}