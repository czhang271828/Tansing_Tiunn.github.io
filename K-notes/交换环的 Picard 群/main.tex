\documentclass{MainStyle}

\usepackage{amsthm, amsfonts, amsmath, amssymb, quiver, mathrsfs, newclude, tikz-cd, ctex}

% Customise href Colours.
\usepackage[colorlinks = true,
            linkcolor = blue,
            urlcolor  = blue,
            citecolor = blue,
            anchorcolor = blue]{hyperref}

\newcommand{\changeurlcolor}[1]{\hypersetup{urlcolor=#1}}       

\newcommand*{\name}{张陈成}
\newcommand*{\id}{023071910029}
\newcommand*{\course}{$K$-理论笔记}
\newcommand*{\assignment}{交换环的 Picard 群}

\theoremstyle{definition}
\newtheorem{example}{例}

\theoremstyle{definition}
\newtheorem{slogan}{原旨}

\theoremstyle{definition}
\newtheorem{definition}{定义}

\theoremstyle{definition}
\newtheorem{proposition}{命题}

\theoremstyle{definition}
\newtheorem{problem}{问题}

\theoremstyle{definition}
\newtheorem{assumption}{假定}

\theoremstyle{definition}
\newtheorem{theorem}{定理}

\theoremstyle{remark}
\newtheorem{remark}{注}

\theoremstyle{remark}
\newtheorem{lemma}{引理}
\allowdisplaybreaks

\begin{document}
\maketitle

\section{Zariski 拓扑简介}

\section{交换环的 Picard 群}

\begin{definition}[投射模的秩函数]
    Kaplansky 定理表明局部环上的投射模自由, 其交换且有限生成之情形已在前文证明(中山引理之推论).\par
    今给定交换环 $R$, 定义投射模 $P$ 的秩函数为
    \begin{align*}
        \mathrm{rank}_P:\mathrm{spec}(R)\to \mathbb N,\quad \mathfrak p\mapsto \mathrm{rank}_{R_{\mathfrak p}}(P_{\mathfrak p}).
    \end{align*}
    简而言之, $\mathrm{rank}_P(\mathfrak p)$ 是 $R$-模 $P$ 在 $\mathfrak p$-局部化下(作为自由 $R_{\mathfrak p}$-模)的秩.
\end{definition}

\begin{proposition}[对偶模回顾]
    给定交换环 $R$, 定义 $X\in \mathsf{Ob}(R\mathrm{-Mod})$ 的对偶模为
    \begin{align*}
        X^\ast:=\mathrm{Hom}_{R\mathrm{-Mod}}(X,R)\quad \in \mathsf{Ob}(R^{\mathrm{op}}\mathrm{-Mod}).
    \end{align*}
    有以下关于对偶模的常用性质.
    \begin{enumerate}
        \item $\varepsilon: P\to P^{\ast\ast}$ 为典范单态射.
        \item 自由模与投射模的一种等价定义如下.
              \begin{itemize}
                  \item $F$ 是\textbf{自由 $R$-模}, 当且仅当存在指标集 $I$ 与 $\{(x_i,f_i)\in F\times F^\ast\}_{i\in I}$ 使得有分解(有限和) $\displaystyle x=\sum _{i\in I}f_i(x)x_i$, 且有限和 $\displaystyle x=\sum _{i\in I}a_i x_i$ 对一切 $x\in F$ \textbf{唯一}.
                  \item $P$ 是\textbf{投射 $R$-模}, 当且仅当存在指标集 $I$ 与 $\{(x_i,f_i)\in P\times P^\ast\}_{i\in I}$ 使得有分解(有限和) $\displaystyle x=\sum _{i\in I}f_i(x)x_i$, 但有限和 $\displaystyle x=\sum _{i\in I}a_i x_i$ 对 $x\in P$ \textbf{不必唯一}.
              \end{itemize}
        \item 有限生成投射模的对偶模同为投射 $R$-模. 具体地, 对任意 $P\oplus Q\simeq R^n$ 总有
              \begin{align*}
                  \mathrm{Hom}_R(P,R)\oplus \mathrm{Hom}_R(Q,R)\simeq \mathrm{Hom}_R(R^n,R)\simeq (\mathrm{End}_R(R))^n\simeq R^n.
              \end{align*}
              此时 $\varepsilon: P^{\ast\ast}\simeq P$ 为同构.
    \end{enumerate}
\end{proposition}

\begin{definition}[$R$-模范畴的半环结构]
    $(R\mathrm{-Mod},\oplus, \otimes)$ 为交换半环, 即,
    \begin{enumerate}
        \item 环中元素为 $\mathsf{Ob}(R\mathrm{-Mod})/\simeq$. 为方便记号, 今后省略商关系.
        \item $(R\mathrm{-Mod},\oplus)$ 为交换幺半群, 其幺元为 $0$;
        \item $(R\mathrm{-Mod}, \otimes)$ 为交换幺半群, 其幺元为 $R$;
        \item $\oplus$ 与 $\otimes$ 分别作为加法与乘法, 满足分配律.
    \end{enumerate}
\end{definition}

\begin{remark}
    函子 $-\otimes M$ 给出范畴 $R\mathrm{-Mod}$ 到自身的范畴等价, 当且仅当 $M$ 是环 $(R\mathrm{-Mod},\oplus,\times)$ 的乘法可逆元. 换言之, 存在 $N$ 使得 $N\otimes M\simeq R\simeq M\otimes N$.
\end{remark}

\begin{definition}[可逆模(线丛)]
    取交换环 $R$ 上有限生成模 $M$. 称 $M$ 可逆, 若以下等价命题成立.
    \begin{enumerate}
        \item 存在 $R$-模 $N$ 使得 $M\otimes N\simeq R$, 且 $M\simeq \mathrm{Hom}_R(N,R)$.
        \item $M\otimes_R-$ 为 $R$-模范畴到自身的等价.
        \item $M$ 是有限生成的秩恒为 $1$ 的投射模.
    \end{enumerate}
    实际上有 $\mathrm{Hom}_R(N,R)\simeq M$.
\end{definition}

\begin{definition}[Picard 群]
    记环 $R$ 中 Picard 群为 $\mathrm{Pic}(R)$ 有限生成可逆模 $\langle M\rangle$ 构成的乘法群. 其中
    \begin{enumerate}
        \item $\langle M\otimes_R N\rangle=\langle M\rangle\cdot \langle  N\rangle$.
        \item $\langle \mathrm{Hom}_R(M,R)\rangle=\langle M\rangle^{-1}$.
        \item $\langle R\rangle$ 为乘法单位.
    \end{enumerate}
\end{definition}

\begin{remark}
    $\mathrm{Pic}:\mathrm{Ring}\to \mathrm{Ab}$ 为(协变)函子. 特别地,
    \begin{align*}
        \mathrm{Pic}:\left[R\overset f\longrightarrow S\right]\mapsto [P\mapsto S\otimes_R P].
    \end{align*}
\end{remark}

\end{document}