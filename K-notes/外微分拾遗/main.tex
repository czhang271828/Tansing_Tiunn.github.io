\documentclass{MainStyle}

\usepackage{amsthm, amsfonts, amsmath, amssymb, quiver, mathrsfs, newclude, tikz-cd, ctex}

% Customise href Colours.
\usepackage[colorlinks = true,
            linkcolor = blue,
            urlcolor  = blue,
            citecolor = blue,
            anchorcolor = blue]{hyperref}

\newcommand{\changeurlcolor}[1]{\hypersetup{urlcolor=#1}}       

\newcommand*{\name}{张陈成}
\newcommand*{\id}{023071910029}
\newcommand*{\course}{$K$-理论笔记}
\newcommand*{\assignment}{外微分拾遗}

\theoremstyle{definition}
\newtheorem{example}{例}

\theoremstyle{definition}
\newtheorem{slogan}{原旨}

\theoremstyle{definition}
\newtheorem{definition}{定义}

\theoremstyle{definition}
\newtheorem{proposition}{命题}

\theoremstyle{definition}
\newtheorem{problem}{问题}

\theoremstyle{definition}
\newtheorem{assumption}{假定}

\theoremstyle{definition}
\newtheorem{theorem}{定理}

\theoremstyle{remark}
\newtheorem{remark}{注}

\theoremstyle{remark}
\newtheorem{lemma}{引理}
\allowdisplaybreaks

\begin{document}
\maketitle
\tableofcontents

\section{模的外积}

\begin{definition}[对称(反对称)函数]
    给定 $R$-模 $M$ 与 $N$. 今取定 $m\in \mathbb N_{\geq 1}$ 以及态射 $M^m\overset f\longrightarrow N$.
    \begin{enumerate}
        \item 称 $f$ 是对称的, 若 $f(x_1,x_2,\ldots ,x_m)=f(x_{\sigma(1)},x_{\sigma(2)},\ldots, x_{\sigma (m)})$ 对一切置换 $\sigma\in S_m$ 成立;
        \item 称 $f$ 是反对称的, 若 $f(x_1,x_2,\ldots ,x_m)=(-1)^\sigma f(x_{\sigma(1)},x_{\sigma(2)},\ldots, x_{\sigma (m)})$ 对一切置换 $\sigma\in S_m$ 成立;
        \item 称 $f$ 是交错的, 若 $f$ 满足以下论断: 若存在 $i\neq j$ 使得 $x_i=x_j$, 则 $f(x_1,x_2,\ldots ,x_m)=0$.
    \end{enumerate}
    显然, 反对称等价于``对换改变符号'', 亦等价于交错.
\end{definition}

\begin{definition}[模的外积]
    给定反对称态射 $M^m\overset f\longrightarrow N$, 则有如下交换图
    % https://q.uiver.app/#q=WzAsNSxbMCwxLCJNXm0iXSxbMSwxLCJcXGJpZ290aW1lcyBebU0iXSxbMiwxLCJcXGJpZ3dlZGdlXm0oTSkiXSxbMSwyLCJOIl0sWzEsMCwiXFxtYXRocm17a2VyfShcXHRpbGRlIGYpIl0sWzAsMSwiXFxwaSJdLFsxLDMsIlxcdGlsZGUgZiIsMix7InN0eWxlIjp7ImJvZHkiOnsibmFtZSI6ImRhc2hlZCJ9fX1dLFsxLDIsImoiLDAseyJzdHlsZSI6eyJoZWFkIjp7Im5hbWUiOiJlcGkifX19XSxbMiwzLCJcXG92ZXJsaW5lIGYiLDAseyJzdHlsZSI6eyJib2R5Ijp7Im5hbWUiOiJkYXNoZWQifX19XSxbMCwzLCJmIiwyXSxbNCwxXV0=
    \[\begin{tikzcd}
            & {\mathrm{ker}(\tilde f)} \\
            {M^m} & {\bigotimes ^mM} & {\bigwedge^m(M)} \\
            & N
            \arrow["\pi", from=2-1, to=2-2]
            \arrow["{\tilde f}"', dashed, from=2-2, to=3-2]
            \arrow["j", two heads, from=2-2, to=2-3]
            \arrow["{\overline f}", dashed, from=2-3, to=3-2]
            \arrow["f"', from=2-1, to=3-2]
            \arrow[from=1-2, to=2-2]
        \end{tikzcd}.\]
    \begin{enumerate}
        \item $\pi :M^m\to \bigotimes^m M$ 由张量积之范性质保证.
        \item 对任意反对称态射 $f\in \mathrm{Hom}_R(M^m,N)$, 总有 $\mathrm{ker}(\tilde f)\subseteq \langle x_1\otimes \cdots \otimes x_n\mid \text{存在 }i\neq j\text{ 使得 }x_i=x_j\rangle =:J$.
    \end{enumerate}
    从而定义外积 $\bigwedge ^m (M):=\dfrac{\bigotimes^m M}J$. 用泛性质语言描述之, 任意 $M^m$ 出发的反对称态射通过 $\bigwedge^m(M)$ 分解.
\end{definition}

\begin{example}
    $\bigwedge^0(M)\simeq \bigotimes^0(M)\simeq R$, $\bigwedge ^1(M)\simeq \bigotimes ^1(M)\simeq M$. 记 $I:=(x,y)$ 为 $R$ 的理想, 则 $\bigwedge^2(I)$ 为主理想.
\end{example}

\begin{proposition}
    若 $M$ 有限生成, 记其极小生成集大小为 $n$. 则 $\bigwedge^m(M)=0$ 对一切 $m>n$ 成立.
\end{proposition}



\end{document}