\documentclass{MainStyle}

\usepackage{amsthm, amsfonts, amsmath, amssymb, quiver, mathrsfs, newclude, tikz-cd}

% Customise href Colours.
\usepackage[colorlinks = true,
            linkcolor = blue,
            urlcolor  = blue,
            citecolor = blue,
            anchorcolor = blue]{hyperref}

\newcommand{\changeurlcolor}[1]{\hypersetup{urlcolor=#1}}       

\newcommand*{\name}{Chencheng Zhang}
\newcommand*{\id}{023071910029}
\newcommand*{\course}{Lecture Notes for $K$-seminar}
\newcommand*{\assignment}{Recollections from Projective Modules}

\theoremstyle{definition}
\newtheorem{example}{Example}

\theoremstyle{definition}
\newtheorem{slogan}{SLOGAN}

\theoremstyle{definition}
\newtheorem{definition}{Definition}

\theoremstyle{definition}
\newtheorem{proposition}{Proposition}

\theoremstyle{definition}
\newtheorem{problem}{Problem}

\theoremstyle{definition}
\newtheorem{assumption}{Assumption}

\theoremstyle{definition}
\newtheorem{theorem}{Theorem}

\theoremstyle{remark}
\newtheorem{remark}{Remark}

\theoremstyle{remark}
\newtheorem{lemma}{Lemma}
\allowdisplaybreaks

\begin{document}
\maketitle
\section{About the Lecture}
We strongly recommend the readers to review the basic definitions and facts about
\begin{enumerate}
    \item groups, rings, modules, field, algebra, etc;
    \item the definition of Abelian category $\mathcal A$;
    \item basic knowledge on categories, such as commutative diagrams.
\end{enumerate}
The Lecture Note mainly discusses
\begin{enumerate}
    \item When free objects are defined and how free objects related to projective objects.
    \item Equivalent definitions of projective modules (for small Abelian categories).
    \item The chain: Free Mods $\to$ Stably Free Mods $\to$ Projective Mods $\to$ Flat Mods $\to$ Torison Free Mods.
\end{enumerate}

\section{Equivalent definitions of projective modules}

\begin{definition}[The \href{https://ncatlab.org/nlab/show/adjoint+functor}{adjoint pair} of Free and Forgetful functors]
    For category $\mathcal C$ and $\mathrm{Set}$ the category of sets, if there exists an adjoint pair $(F\dashv U)$ such that
    \begin{itemize}
        \item $F:\mathrm{Set}\to \mathcal C$ is the \href{https://ncatlab.org/nlab/show/free+functor}{free functor} sending the category of sets to $\mathcal C$;
        \item $U$ is the \href{https://ncatlab.org/nlab/show/forgetful+functor}{forgetul functor} sending the collection of objects (resp., morphisms) of $\mathcal C$ to the underlying sets (resp., set mappings);
        \item the collection $\{\varepsilon_X:FU(X)\to X\}_{X\in \mathsf{Ob}(\mathcal C)}$ is called \href{https://ncatlab.org/nlab/show/unit+of+an+adjunction}{co-unit};
        \item $\{\eta_X:S\to UF(X)\}_{S\in\mathsf{Ob}(\mathrm{Set})}$ is called \href{https://ncatlab.org/nlab/show/unit+of+an+adjunction}{unit}.
    \end{itemize}
\end{definition}

\begin{remark}
    \href{https://ncatlab.org/nlab/show/Freyd-Mitchell+embedding+theorem}{Freyd-Mitchell embedding theorem} says that every small Abelian category $\mathcal A$ is a full subcategory of a category of modules over some ring $R$, such that the embedding functor $\mathcal A\hookrightarrow R\mathrm{-Mod}$ is an exact functor.
\end{remark}

\begin{slogan}
    Small Abelian categories $Longleftarrow R\mathrm{-Mod}$ categories.
\end{slogan}

\begin{definition}[projective modules]
    Suppose that $\mathcal A$ is an Abelian category. We say $P\in \mathcal A$ is projective, whenever
    \begin{itemize}
        \item for arbitrary epimorphism $X\overset \pi\twoheadrightarrow Y$ and $P\overset f\to Y$, there exists a lift $\tilde f$ such that the following diagram commutes
              % https://q.uiver.app/#q=WzAsNCxbMSw0XSxbMSwwLCJQIl0sWzEsMSwiWSJdLFswLDEsIlgiXSxbMywyLCJcXHBpIiwyLHsic3R5bGUiOnsiaGVhZCI6eyJuYW1lIjoiZXBpIn19fV0sWzEsMywiXFx0aWxkZSBmIiwyLHsic3R5bGUiOnsiYm9keSI6eyJuYW1lIjoiZGFzaGVkIn19fV0sWzEsMiwiZiJdXQ==
              \[\begin{tikzcd}
                      & P \\
                      X & Y \\
                      \\
                      \\
                      & {}
                      \arrow["\pi"', two heads, from=2-1, to=2-2]
                      \arrow["{\tilde f}"', dashed, from=1-2, to=2-1]
                      \arrow["f", from=1-2, to=2-2]
                  \end{tikzcd}.\]
    \end{itemize}
\end{definition}

\begin{proposition}
    Try to update
\end{proposition}

\end{document}